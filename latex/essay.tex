\documentclass[a4paper]{article}

\def\npart{III Essay}

\def\ntitle{Walking Deeper on Dynamic Graphs}

\def\ndate{\today}

\ifx \nauthor\undefined
  \def\nauthor{Qiangru Kuang}
\else
\fi

\ifx \ntitle\undefined
  \def\ntitle{Template}
\else
\fi

\ifx \nauthoremail\undefined
  \def\nauthoremail{qk206@cam.ac.uk}
\else
\fi

\ifx \ndate\undefined
  \def\ndate{\today}
\else
\fi

\title{\ntitle}
\author{\nauthor}
\date{\ndate}

%\usepackage{microtype}
\usepackage{mathtools}
\usepackage{amsthm}
\usepackage{stmaryrd}%symbols used so far: \mapsfrom
\usepackage{empheq}
\usepackage{amssymb}
\let\mathbbalt\mathbb
\let\pitchforkold\pitchfork
\usepackage{unicode-math}
\let\mathbb\mathbbalt%reset to original \mathbb
\let\pitchfork\pitchforkold

\usepackage{imakeidx}
\makeindex[intoc]

%to address the problem that Latin modern doesn't have unicode support for setminus
%https://tex.stackexchange.com/a/55205/26707
\AtBeginDocument{\renewcommand*{\setminus}{\mathbin{\backslash}}}
\AtBeginDocument{\renewcommand*{\models}{\vDash}}%for \vDash is same size as \vdash but orginal \models is larger
\AtBeginDocument{\let\Re\relax}
\AtBeginDocument{\let\Im\relax}
\AtBeginDocument{\DeclareMathOperator{\Re}{Re}}
\AtBeginDocument{\DeclareMathOperator{\Im}{Im}}
\AtBeginDocument{\let\div\relax}
\AtBeginDocument{\DeclareMathOperator{\div}{div}}

\usepackage{tikz}
\usetikzlibrary{automata,positioning}
\usepackage{pgfplots}
%some preset styles
\pgfplotsset{compat=1.15}
\pgfplotsset{centre/.append style={axis x line=middle, axis y line=middle, xlabel={$x$}, ylabel={$y$}, axis equal}}
\usepackage{tikz-cd}
\usepackage{graphicx}
\usepackage{newunicodechar}

\usepackage{fancyhdr}

\fancypagestyle{mypagestyle}{
    \fancyhf{}
    \lhead{\emph{\nouppercase{\leftmark}}}
    \rhead{}
    \cfoot{\thepage}
}
\pagestyle{mypagestyle}

\usepackage{titlesec}
\newcommand{\sectionbreak}{\clearpage} % clear page after each section
\usepackage[perpage]{footmisc}
\usepackage{blindtext}

%\reallywidehat
%https://tex.stackexchange.com/a/101136/26707
\usepackage{scalerel,stackengine}
\stackMath
\newcommand\reallywidehat[1]{%
\savestack{\tmpbox}{\stretchto{%
  \scaleto{%
    \scalerel*[\widthof{\ensuremath{#1}}]{\kern-.6pt\bigwedge\kern-.6pt}%
    {\rule[-\textheight/2]{1ex}{\textheight}}%WIDTH-LIMITED BIG WEDGE
  }{\textheight}% 
}{0.5ex}}%
\stackon[1pt]{#1}{\tmpbox}%
}

%\usepackage{braket}
\usepackage{thmtools}%restate theorem
\usepackage{hyperref}

% https://en.wikibooks.org/wiki/LaTeX/Hyperlinks
\hypersetup{
    %bookmarks=true,
    unicode=true,
    pdftitle={\ntitle},
    pdfauthor={\nauthor},
    pdfsubject={Mathematics},
    pdfcreator={\nauthor},
    pdfproducer={\nauthor},
    pdfkeywords={math maths \ntitle},
    colorlinks=true,
    linkcolor={red!50!black},
    citecolor={blue!50!black},
    urlcolor={blue!80!black}
}

\usepackage{cleveref}



% TODO: mdframed often gives bad breaks that cause empty lines. Would like to switch to tcolorbox.
% The current workaround is to set innerbottommargin=0pt.

%\usepackage[theorems]{tcolorbox}





\usepackage[framemethod=tikz]{mdframed}
\mdfdefinestyle{leftbar}{
  %nobreak=true, %dirty hack
  linewidth=1.5pt,
  linecolor=gray,
  hidealllines=true,
  leftline=true,
  leftmargin=0pt,
  innerleftmargin=5pt,
  innerrightmargin=10pt,
  innertopmargin=-5pt,
  % innerbottommargin=5pt, % original
  innerbottommargin=0pt, % temporary hack 
}
%\newmdtheoremenv[style=leftbar]{theorem}{Theorem}[section]
%\newmdtheoremenv[style=leftbar]{proposition}[theorem]{proposition}
%\newmdtheoremenv[style=leftbar]{lemma}[theorem]{Lemma}
%\newmdtheoremenv[style=leftbar]{corollary}[theorem]{corollary}

\newtheorem{theorem}{Theorem}[section]
\newtheorem{proposition}[theorem]{Proposition}
\newtheorem{lemma}[theorem]{Lemma}
\newtheorem{corollary}[theorem]{Corollary}
\newtheorem{axiom}[theorem]{Axiom}
\newtheorem*{axiom*}{Axiom}

\surroundwithmdframed[style=leftbar]{theorem}
\surroundwithmdframed[style=leftbar]{proposition}
\surroundwithmdframed[style=leftbar]{lemma}
\surroundwithmdframed[style=leftbar]{corollary}
\surroundwithmdframed[style=leftbar]{axiom}
\surroundwithmdframed[style=leftbar]{axiom*}

\theoremstyle{definition}

\newtheorem*{definition}{Definition}
\surroundwithmdframed[style=leftbar]{definition}

\newtheorem*{slogan}{Slogan}
\newtheorem*{eg}{Example}
\newtheorem*{ex}{Exercise}
\newtheorem*{remark}{Remark}
\newtheorem*{notation}{Notation}
\newtheorem*{convention}{Convention}
\newtheorem*{assumption}{Assumption}
\newtheorem*{question}{Question}
\newtheorem*{answer}{Answer}
\newtheorem*{note}{Note}
\newtheorem*{application}{Application}

%operator macros

%basic
\DeclareMathOperator{\lcm}{lcm}

%matrix
\DeclareMathOperator{\tr}{tr}
\DeclareMathOperator{\Tr}{Tr}
\DeclareMathOperator{\adj}{adj}

%algebra
\DeclareMathOperator{\Hom}{Hom}
\DeclareMathOperator{\End}{End}
\DeclareMathOperator{\id}{id}
\DeclareMathOperator{\im}{im}
\DeclareMathOperator{\coker}{coker}
\DeclarePairedDelimiter{\generation}{\langle}{\rangle}

%groups
\DeclareMathOperator{\sym}{Sym}
\DeclareMathOperator{\sgn}{sgn}
\DeclareMathOperator{\inn}{Inn}
\DeclareMathOperator{\aut}{Aut}
\DeclareMathOperator{\GL}{GL}
\DeclareMathOperator{\SL}{SL}
\DeclareMathOperator{\PGL}{PGL}
\DeclareMathOperator{\PSL}{PSL}
\DeclareMathOperator{\SU}{SU}
\DeclareMathOperator{\UU}{U}
\DeclareMathOperator{\SO}{SO}
\DeclareMathOperator{\OO}{O}
\DeclareMathOperator{\PSU}{PSU}
\DeclareMathOperator{\Sp}{Sp}


%hyperbolic
\DeclareMathOperator{\sech}{sech}

%field, galois heory
\DeclareMathOperator{\ch}{ch}
\DeclareMathOperator{\gal}{Gal}
\DeclareMathOperator{\emb}{Emb}



%ceiling and floor
%https://tex.stackexchange.com/a/118217/26707
\DeclarePairedDelimiter\ceil{\lceil}{\rceil}
\DeclarePairedDelimiter\floor{\lfloor}{\rfloor}


\DeclarePairedDelimiter{\innerproduct}{\langle}{\rangle}

%\DeclarePairedDelimiterX{\norm}[1]{\lVert}{\rVert}{#1}
\DeclarePairedDelimiter{\norm}{\lVert}{\rVert}



%Dirac notation
%TODO: rewrite for variable number of arguments
\DeclarePairedDelimiterX{\braket}[2]{\langle}{\rangle}{#1 \delimsize\vert #2}
\DeclarePairedDelimiterX{\braketthree}[3]{\langle}{\rangle}{#1 \delimsize\vert #2 \delimsize\vert #3}

\DeclarePairedDelimiter{\bra}{\langle}{\rvert}
\DeclarePairedDelimiter{\ket}{\lvert}{\rangle}




%macros

%general

%divide, not divide
\newcommand*{\divides}{\mid}
\newcommand*{\ndivides}{\nmid}
%vector, i.e. mathbf
%https://tex.stackexchange.com/a/45746/26707
\newcommand*{\V}[1]{{\ensuremath{\symbf{#1}}}}
%closure
\newcommand*{\cl}[1]{\overline{#1}}
%conjugate
\newcommand*{\conj}[1]{\overline{#1}}
%set complement
\newcommand*{\stcomp}[1]{\overline{#1}}
\newcommand*{\compose}{\circ}
\newcommand*{\nto}{\nrightarrow}
\newcommand*{\p}{\partial}
%embed
\newcommand*{\embed}{\hookrightarrow}
%surjection
\newcommand*{\surj}{\twoheadrightarrow}
%power set
\newcommand*{\powerset}{\mathcal{P}}

%matrix
\newcommand*{\matrixring}{\mathcal{M}}

%groups
\newcommand*{\normal}{\trianglelefteq}
%rings
\newcommand*{\ideal}{\trianglelefteq}

%fields
\renewcommand*{\C}{{\mathbb{C}}}
\newcommand*{\R}{{\mathbb{R}}}
\newcommand*{\Q}{{\mathbb{Q}}}
\newcommand*{\Z}{{\mathbb{Z}}}
\newcommand*{\N}{{\mathbb{N}}}
\newcommand*{\F}{{\mathbb{F}}}
%not really but I think this belongs here
\newcommand*{\A}{{\mathbb{A}}}

%asymptotic
\newcommand*{\bigO}{O}
\newcommand*{\smallo}{o}

%probability
\newcommand*{\prob}{\mathbb{P}}
\newcommand*{\E}{\mathbb{E}}

%vector calculus
\newcommand*{\gradient}{\V \nabla}
\newcommand*{\divergence}{\gradient \cdot}
\newcommand*{\curl}{\gradient \cdot}

%logic
\newcommand*{\yields}{\vdash}
\newcommand*{\nyields}{\nvdash}

%differential geometry
\renewcommand*{\H}{\mathbb{H}}
\newcommand*{\transversal}{\pitchfork}
\renewcommand{\d}{\mathrm{d}} % exterior derivative

%number theory
\newcommand*{\legendre}[2]{\genfrac{(}{)}{}{}{#1}{#2}}%Legendre symbol

%algebraic geometry
\DeclareMathOperator{\Spec}{Spec}
\DeclareMathOperator{\Proj}{Proj}

\let\SO\undefined
\usepackage{tkz-graph}
\usepackage{thm-restate}

\newcommand{\shadow}{\partial}
\renewcommand{\P}{\mathbb P}
\renewcommand{\E}{\mathbb E}
\newcommand{\D}{\mathcal D}
\newcommand{\rar}{\overrightarrow r}
\newcommand{\lar}{\overleftarrow r}
\begin{document}

\begin{titlepage}
  \begin{center}
%    \includegraphics[width=0.6\textwidth]{logo.jpg}\par
    \vspace{2cm}
    {\scshape\huge University of \par
      \Huge Cambridge \par}
    \vspace{1cm}
    {\scshape\huge Mathematics Tripos \par}
    \vspace{2cm}
    {\huge Part \npart \par}
    \vspace{0.6cm}
    {\Huge \bfseries \ntitle \par}
    \vspace{1.2cm}
    {\Large\ndate \par}
    \vspace{2cm}
    
    {\large \emph{Written by } \par}
    \vspace{0.2cm}
    {\Large \scshape Joshua Snyder}
 \end{center}
\end{titlepage}

\tableofcontents

\section{Motivation}
The motivation for studying nodes in graphs and their representations comes from the desire to understand networks of people or objects and their relations. The motivating question for this essay is
\begin{question}[Motivating Question]
  Given a large network of people how can we quantify the relationships between them?
\end{question}
The natural way to go about this is to let nodes represent persons and let edges
between them represent connections from which we can induce some understanding
of relationship or trust between two people. This task is difficult; even with a
relatively small number of people it is not a task on which humans perform very
well and the task rapidly becomes difficult as the number of nodes in the graph increases.\\
Many of the networks which we wish to study are dynamic; that is they change
with time. In a large network it is common that small changes occur during each
epoch of time that over time cause larger changes to the network structure. How
do we quantify these changes and their impact on the relationships in the
network without having to completely re-analyse the network after ever epoch of
time, losing the information that we already learned. A frequent example upon
which we can draw similarity is that of social networks. A large social network
has new users (nodes) being added and new relationships (edges) formed in each
epoch of time, however in any given short time period the graph representing the
social network does not change substantially. It would be both foolish and
costly to re-analyse the graph after each epoch however most of the previous
literature has focused on static graphs. In the latter half of this essay I will
give an account of recent progress into the application of DeepWalk and similar
algorithms to dynamic graphs.

\section{Summary}
There are three main objectives of this essay:
\begin{itemize}
\item Firstly the essay will give a mathematical outline of the DeepWalk
  algorithm and it's application to social representation learning. We will
  briefly discuss the benefits and drawbacks of the algorithm.
\item Secondly we will look at an implementation of DeepWalk to dynamic
  graphs. The challenge here is to develop an unbiased representation of a
  graph at time $t+1$ given it's representation at time $t$, without
  re-analysing the entire network. This is a very important task since many
  networks, especially social ones, are constantly changing. However in any given
  epoch of time the network structure is unlikely to undergoe dramatic change and
  so a computationally effective algorithm will not re-analyse the network at
  each step.
\item Thirdly, we will exhibit an implementation of the outlined dynamic DeepWalk
  application to a social data set [which data set?] and give suggestions as
  to good applications of Dynamic DeepWalking. [talk here about the
  application once I have found one]
\end{itemize}

The section on DeepWalk explores the fundamental connection between DeepWalk and
matrix factorisation as presented in the paper by Qiu et al.\cite{qiu2018}. In
this section, we demonstrate the conditions under which DeepWalk is factorising
an appropriate matrix that stores some measure of similarity between
nodes in the graph. In the section ``Why DeepWalk sucks'' we look at some of the
criticism that followed the original DeepWalk paper and how these criticisms
were addressed by future authors, the section is indicative of the shortcomings
of DeepWalk but is by no means exhaustive.\\
After this, the section ``Dynamic DeepWalking'' focusses on applying a variant of the DeepWalk algorithm, proposed
by Sajjad et al. \cite{sajjad2019}, for dynamic graphs. This is especially
important in the context of social graphs, which usually contain information
about the interactions of individuals over time. Learning social
representations for dynamic graphs can allow for a better understanding of how
communities move and change over time.

\section{DeepWalk}
This section gives an outline of the social representation learning algorithm
DeepWalk, first introduced in the seminal paper DeepWalk: Online Learning of Social
Representations by B. Perozzi et. al. \cite{deepwalk}. The method proposed in
this paper not only demonstrated performance improvements from previous methodologies but
also motivated an entirely different approach. At the time of the paper being
written, significant advancements were being made in natural languade processing (NLP)
and the idea of word embeddings was becoming popular through an embedding
algorithm known as word2vec \cite{mikolov2013efficient,mikolov2013distributed}.
DeepWalk implements this algoritm but replaces the idea of the context of a word in
a sentence with the context of a node in a random walk on a graph. This is the
crucial concept of DeepWalk from which the remaining details naturally follow.

The original paper on DeepWalk is lacking in a mathematical underpinning and in
this section we will model the algorithm mathematically. It is suggested that the reader
is familiar with the concepts outlined in the paper by Perozzi et. al. prior to
reading this (more) mathematical exposition. I have endevoured to use similar
notation to the original paper to ease cross-referencing. Without further ado, let us begin
our journey.

\begin{definition}
  Let $G = (V, E)$ be an undirected graph (representing a network). $V$ represents the
  members of the network, commonly referred to as the nodes and $E \subset V
  \times V$ represents their connections.\\

  The nodes and edges have lables and $G_L = (V, E, X, Y)$ represents the
  partially labelled network. $X \in \R^{|V| \times S}$ where $S$ is the size of
  the feature space for each attribute vector and $Y \in \R^{|V| \times
    |\mathcal{Y}|}$, where $\mathcal{Y}$ is the set of labels.
\end{definition}

Our goal is to learn $X_E \in \R^{|V| \times d}$ where $d$ is a small number
of latent dimensions. The idea is that each latent dimension contributes a 
dimensional...

*** Explain the DeepWalk algorithm here ***

\subsection{Introduction to SkipGram}
To understand DeepWalk mathematically, we first need to understand what the SkipGram
model is doing, since this is the model underpinning DeepWalk, which took it
from NLP and applied it to graphs. In the context of graph networks, SkipGram trains a neural network to do the
following task:\\
Given an input vertex $v$ and a random walk, $W_{v}$, of length $2t+1$ with $v$
at it's centre. Pick a nearby vertex at random. The task of the neural network is to
predict the probability that each vertex in $V$ will be the randomly chosen
vertex. Therefore, verticies far away on the graph that correspond to unfamiliar
nodes are unlikely to co-occur on the same random walk and will be assigned a
low probability. Conversely, nearby and well connected vertices are likely to
co-occur on a random walk with input $v$ and thus will be assigned higher
probabilities. This allows us to train a network with weights that represent the
connectedness between nodes on the graph. The neural network is trained by
feeding it pairs of nodes $(v, c)$ where $v$ represents the input node and $c$ is a context node, which lies within distance
$t$ of the vertex $v$.\\ 

To formalise this, each of the nodes $v \in V$ are represented by a one-hot
encoding vector $e_v \in \mathbb{R}^{|V|}$ allowing us to feed $e_v$ into the
neural network. When $e_v$ is fed into the network, a single linear hidden layer with
$d$ neurons is used, where $d$ is the desired dimension of the latent
representations, which is then passed to a softmax classifier for output. The
output of the network is a vector $o \in \mathbb{R}^d$ containing the estimated
probabilities that a randomly selected nearby word is that vocabulary word.

The idea behind having a linear hidden layer, which does not use an activation
function, is to use the resulting weight matrix $W \in \mathbb{R}^{|V| \times
  d}$ as the embedding vectors for the nodes in the graph. This is intuitive as
the hidden layer acts as a bottleneck that tries to represent as much
information as possible to distinguish the nodes, but is only allowed $d$
neurons to do so. Since $d \ll |V|$ there is a low risk of overfitting.

*** Is our vocabulary the whole graph, or is it just the random walk that we
feed to the SkipGram model, perhaps it is just the random walk, be careful with
this! ***

*** Create here an image resembling the one found in the blog post on SkipGram
but for graphs and with $d$ dimensions ***

The algorithm used in DeepWalk varies slightly from the SkipGram algorithm
discussed. Calculating the normalization factor in the Softmax layer requires a
computational complexity of $O(|V|)$, in the original DeepWalk paper this is
reduced by using Hierarchical Softmax to approximate the softmax probabilities,
requiring a complexity of only $O(log|V|)$. In particular, a Huffman coding is
used to reduce the access time of frequent elements in the tree, as suggested by
Mikolov et al. in the original Word2Vec
papers.\cite{mikolov2013distributed,mikolov2013efficient}

In later adaptations of DeepWalk, SkipGram with Negative Sampling (SGNS) is used instead of
Hierarchical Softmax. In the remainder of this essay, when referring to DeepWalk, it will be
implicitly assumed that Negative Sampling is used as appose to Hierarchical
Softmax. This is because SGNS has been found to be more efficient
and therefore has been adopted by much of the further literature. This convention will also
serve us well when we look at applying DeepWalk to dynamic graphs since here
Negative Sampling is also applied.\\

*** Can give a summary of NS here, if so just summarise what is said in Blog
Post 2 on SkipGram ***
*** Has NS been shown, or just found empirically, to be more efficient then
Hierarchical Softmax? Check node2vec paper for information on this. ***

\subsection{SkipGram as matrix factorisation}
In this section we will exhibit a proof that SGNS is
equivalent to factorising a certain matrix $M$ into two smaller matricies $W$
and $C$ where the rows in $W$ correspond to the learned embedding of each vertex. This result was first proved by Levy and
Goldberg\cite{levy&goldberg} in the context of word embeddings.\\
\subsubsection{Objective of SGNS}
Given an arbitrary input-context pair $(v,c)$ the objective is to determine if
the pair comes from the random-walk corpus $\mathcal{D}$.\\
Let $P(\D = 1 | v, c)$ denote the probability that $(v,c)$ comes from a random
walk on the graph and $P(\D = 0| v, c)$
the probability it does not. Then the distribution is modelled by a signoid
function
\[P(D = 1 | v, c) = \sigma(\vec{v} \cdot \vec{c}) = \frac{1}{1 + e^{-\vec{v} \cdot \vec{c}}}\]
where $\vec{v}$ and $\vec{c}$ are $d$-dimensional vectors to be learned. SGNS attempts to maximise $P(\mathcal{D} = 1 | v,c)$ for observed pairs $(v, c)$
whilst simultaneously maximising $P(\D = 0 | v, c)$ for randomly sampled
negative examples.\\
It assumes that randomly selecting a context $c$ for a given
node $v$ is likely to result in an unobserved pair $(v,c)$. In the context of
social networks, this assumption is reasonable since social networks are almost
always sparse (The number of edges is usually $O(|V|)$). However in a different
context, if the network is dense, then this may be an unreasonable assumption.\\

According to this assumption, the objective function of SGNS for a single
observation $(v,c)$ is:

\[\log{\sigma(\vec{v} \cdot \vec{c})} + b \cdot \E_{c_{N} \sim P_D}\log{\sigma(-\vec{v} \cdot \vec{c})}\]
where the minus sign comes from the fact that $1 - \sigma(x) = \sigma(-x)$, $b$
is the number of negative samples and $c_N$ is the sampled context node, drawn
according to $P_D (c) = \frac{\#(c)}{| \D |}$ which is known as the unigram
distribution.\\

\begin{notation} $\#(v,c)$, $\#(v)$ and $\#(c)$ denote the number of times vertex-context pair
  $(v,c)$, vertex $v$ and context $c$ appear in the generated random-walk corpus
  $\mathcal{D}$ respectively.
\end{notation}

This objective function is trained using stochastic gradient descent with
updates after each  observed pair in the random-walk corpus $\D$. The resulting
global objective becomes
\begin{equation}
  l = \sum_{v \in V} \sum_{c \in V}\#(v,c)\log{\sigma(\vec{v} \cdot \vec{c})} +
  b \cdot \E_{c_{N} \sim P_D}\log{\sigma(-\vec{v} \cdot \vec{c})}
\end{equation}

\subsubsection{Finding the similarity function learned by SkipGram}

If we let $W$ be the matrix with rows $v_i$ (The matrix $W$ is used to highlight
that it is the weight matrix in the neural network previosly described) and $C$ the matrix with columns
$c_i$ then SGNS can be interpreted as factorising a matrix $M = WC^T$. An entry in the matrix $M_{ij}$ corresponds to the dot product $\vec{v_i} \cdot
\vec{c_j}$. Therefore SGNS is factorising a matrix in which each row corresponds
to an input node $v_i \in |V|$ and each column to a context node $c_j \in |V|$ and the value of $M_{ij}$ expresses the
strength of association between the input-context pair $(v_i, c_j)$ using some similarity
function $s(v_i,c_j)$. 
\begin{theorem}[Levy, Goldberg (2014)]
  SkipGram with Negative Sampling (SGNS) is implicitly factorising a matrix $M =
  WC^T$ with
  \[M_{ij} = \log{\frac{\#(v_i,c_j)|\mathcal{D}|}{\#(v_i)\#(c_j)}} - \log{b}\]
  where $W, C \in \mathbb{R}^{|V| \times d}$, and $b$ is the number of negative samples.
\end{theorem}

\begin{proof}
  Firstly, for sufficiently large dimensionality $d$ (so as to allow for a perfect
  reconstruction of $M$), each of the products $v_i \cdot c_j$ can be assumed to
  take their values independently of the others.
  *** WHY. EXPLAIN THIS. Why not sufficiently large value of $|\D|$, why does $d$
  matter here? ***

  Due to this independence, the objective function $l$ can be maximised with
  respect to each pair $v \cdot c$ individually.

  The expectation term in $l$ can be written explicitly as
  \begin{align*}
    \E_{\tilde{c} \sim P_{\D}}[\log{\sigma(-\vec{v} \cdot \vec{\tilde{c}})}] &= \sum_{\tilde{c} \in V}{\frac{\#(\tilde{c})}{|\D|} \log{\sigma(-\vec{v} \cdot \vec{\tilde{c}})}}\\
                                                                             &= \frac{\#(c)}{|\D|} \log{\sigma(-\vec{v} \cdot \vec{c})} + \sum_{\tilde{c} \in V \setminus \{c\}}{\frac{\#(\tilde{c})}{|\D|} \log{\sigma(-\vec{v} \cdot \vec{\tilde{c}})}}
  \end{align*}
  and the objective function $l$ can be expressed as
  \begin{equation*}
    l =  \sum_{v \in V}\sum_{c \in V}\#(v, c)\log{\sigma(\vec{v} \cdot \vec{c})} + \sum_{v \in V}\#(v)\left(b \cdot \E_{\tilde{c} \sim P_{\D}}[\log{\sigma(-\vec{v} \cdot \vec{\tilde{c}})}] \right)
  \end{equation*}
  where the second term comes from the fact that $\#(v) = \sum_{c \in V}\#(v,c)$
  by definition.\\
  Combining these equations gives that the local objective for an input-context
  pair is
  \begin{equation}
    l(v, c) = \#(v, c)\log{\sigma(\vec{v} \cdot \vec{c})} + b \cdot \#(v)\cdot \frac{\#(c)}{|\D|}\log{-\sigma(-\vec{v} \cdot \vec{c})} 
  \end{equation}
  Now to simplify the notation let $x = \vec{v} \cdot \vec{c}$ and, since we are
  assuming each of the $\vec{v_i} \cdot \vec{c_j}$ to take their values
  independently, we take the partial derivatve with respect to $x$ and optimise
  the local objective:
  \[\frac{\partial{l}}{\partial{x}} = \#(v, c) \cdot \log{\sigma(-x)} - b \cdot
    \#(v) \cdot \frac{\#(c)}{|\D|} \cdot \sigma(x)\]
  Where the derivatives are since $\frac{d}{dx} \sigma(x) = \sigma(-x)$. Setting
  the derivative to zero and multipying through by
  $\frac{-e^x}{\sigma(x)\sigma(-x)}$ gives:
  \[\frac{b \cdot \#(v) \cdot \#(c)}{|\D|}e^{2x} + \left( \frac{b \cdot \#(v)
        \cdot \#(c)}{|\D|} - \#(v, c) \right)e^x - \#(v, c) = 0\]
  This is a quadratic equation in $e^x$ with two solutions. The first solution,
  $e^x = -1$ is infeasable since $x \in \R$ and so the appropriate solution is
  \[e^x = \frac{\#(v,c) \cdot |\D|}{\#(v)\#(c)} \cdot \frac{1}{b}\]
  and substituting $x = \vec{v} \cdot \vec{c}$ back into the equation gives
  \[M_{ij} = \vec{v} \cdot \vec{c} = log{\left( \frac{\#(v, c) \cdot |\D|}{\#(v) \cdot
          \#(c)} \right)} - \log{b}\]
  
\end{proof}

Most interestingly, the resulting expression for the similarity function $s$ is
the pointwise mutual information (PMI) shifted by a factor of $\log b$. PMI was
introduced as a measure of association between words in 1990 by Church and Hanks
\cite{church1990} and became widely adopted for NLP tasks.\\

There is an equivalent theorem for SkipGram with Softmax that was proved by Yang
et al.\cite{yangalternative2015} and was later used in their development of
text-associated DeepWalk (TADW)\cite{yang2015} which encorporates text features
of the verticies in a social graph.

\begin{theorem}[Yang et al. (2015)]
  SkipGram with Softmax is implicitly factorising the matrix $M = WC^T$ with
  \[M_{ij} = \log{\frac{\#(v_i,v_j)}{\#(v_i)}}\]
\end{theorem}

The proof of this follows very similarly to the previous theorem and will not be
shown here since we will only be concerned with SGNS.



*** At some point I need to actually outline these algorithms as the other
papers have done, there needs to be an explicit DeepWalk algorithm ***

\subsection{DeepWalk as matrix factorisation}

We continue to look at DeepWalk in the context of matrix factorisation. Much of the proceeding analysis was exhibited in a recent paper by Qiu et
al.\cite{qiu2018} published in 2018 building upon work by Yang et al.\cite{yang2015}
from 2015. The aim of the former paper was to lay the foundationds for, and
unify, the SkipGram based network embedding methods.\\

First we give some preliminary definitions.

\begin{definition}[Adjacency Matrix (A)]
  The adjacency matrix $A \in \R^{|V| \times |V|}$ for a graph $G$ is the matrix with $A_{ij} = 1$ if $(v_i, v_j) \in E(G)$ and $A_{ij} = 0$ otherwise.
\end{definition}

\begin{definition}[Degree Matrix (D)]
  The degree matrix $D \in \R^{|V| \times |V|}$ for a graph $G$ is a diagonal
  matrix with $D_{ii} = d_i = degree(v_i)$ for $v_i \in V$ and $D_{ij} = 0$ otherwise.
\end{definition}

\begin{definition}[Transition Matrix (P)]
  The transition matrix $P \in \R^{|V| \times |V|}$ for a graph $G$ is the
  matrix $P = D^{-1}A$. It has entries $P_{ij} = \frac{1}{d_i}$ if $(v_i, v_j)
  \in E(G)$ and $P_{ij} = 0$ otherwise. It is the transition matrix
  corresponding to a simple random walk on the graph $G$. 
\end{definition}

In their paper, Qiu et al. gave a theoretical understanding of the DeepWalk
algorithm by proving the following theorem:

\begin{restatable}[DeepWalk as implicit matrix factorisation]{theorem}{MainDeepWalk}
  As $t \to \infty$, DeepWalk is equivalent to factorising
  \[\log{\left(\frac{2|E|}{w}\left( \sum_{r = 1}^w P^r  \right) D^{-1}
      \right)} - \log{b}\]
  where $b$ is the negative sampling rate.
\end{restatable}

The theorem assumes that the graph is undirected and that it is connected so
that $P$ is irreducible. The theorem also assumes that the graph is
non-bipartite to ensure that the random walk converges to it's invariant distribution. In the application to
social network graphs, this
assumption will almost certainly hold as the existence of an odd cycle is
expected (Social networks usually contain a large amount of triangles,
representing mutual friends or connections). It is possible however that the
graph is not connected, in this case a dummy node can be introduced which
contains edges to all nodes which will not affect the community structure,
provided the graph sub-communities are sufficiently dense.
What follows is a careful outline of this proof.\\

*** If want to include details on bipartite asssumption, see the folder for
DeepWalk for more information. ***\\

Then $\pi_i = \frac{d_i}{2|E|}$ satisfies the detailed balance equations:
\[\pi_i P_{ij} = d_i\cdot \frac{1}{d_i} = d_j \cdot \frac{1}{d_j} = \pi_j P_{ji}\]
and $\sum_{v_i \in V} \pi_i = 1$.Thus $\pi$ defines a distribution which is
invariant for the simple random walk on the graph. Since the state space is
finite and by assumption, $P$ is irreducible, a random walk on $G$ defines an
irreducible Markov Chain $X$ with transition matrix $P$. Therefore $\pi$ is
unique by the following theorem
\begin{theorem}
  Consider an irreducible Markov chain. Then
  \begin{itemize}
  \item[(i)] There exists an invariant distribution if and only if some state is
    positive recurrent.
  \item[(ii)] If there is an invariant distribution $\pi$, then every state is
    positive recurrent, and
    \[\pi_i = \frac{1}{\mu_i}\]
    for $i \in S$, where $\mu_i$ is the mean recurrence time of $i$. In
    particular, $\pi$ is unique.
  \end{itemize}
\end{theorem}

This is a standard proof in any course on Markov Chains and so the proof is
omitted here. See Page 31 of \cite{markov_chains} for details of a proof.

To proceed with the first important preliminary lemma, one can proceed by
partitioning the random-walk corpus as follows.

\begin{definition}
  For $r = 1, \dots, t$, we define the following
  \[\D_{\rar} = \{ (v, c) : (v, c) \in \D, v = v_j, c = v_{j+r}\}\]
  \[\D_{\lar} = \{ (v, c) : (v, c) \in \D, v = v_{j+r}, c = v_{j}\}\]
  *** Need to give a proper definition to clarify what I mean here ***
  Thus $\D_{\rar}$/$\D_{\lar}$ are sub-multisets of $\D$ such that the context
  $c$ is $r$ steps after or before the vertex $v$ in random walks respectively.
\end{definition}

As an extension of previous definitions, we let $\#(v, c)_{\rar}$ and $\#(v,
c)_{\lar}$ denote the number of times that an input-context pair $(v,c)$ appears
in $\D_{\rar}$ and $\D_{\lar}$ respectively. Then the following lemma holds
\begin{lemma}
  As $t \to \infty$, we have
  \[\frac{\#(v, c)_{\rar}}{|\D_{\rar}|} \overset{p}{\to} \pi_v(P^r)_{v,c} \  \text{and}
    \ \frac{\#(v, c)_{\lar}}{|\D_{\lar}|} \overset{p}{\to} \pi_v(P^r)_{v,c} \]
\end{lemma}
\begin{proof}
***
A proof of this can be found in Paper 2 but the proof is not nice, perhaps I can
come up with an original proof of this using convergence of the random walk to
it's invariant distribution.

Note that by reversibility the two statements should be provably equivalent and
I have used detailed balance here to prove that the two limits given in Paper 2
are the same.

I will leave this to come back to for a nicer proof.
***
\end{proof}

From this we can show that

\begin{lemma}
  When $t \to \infty$, we have
  \[\frac{\#(w, c)}{|\D|} \overset{p}{\to} \frac{1}{w} \sum_{r = 1}^w \pi_v
    (P^r)_{v,c}\]
\end{lemma}
\begin{proof}
  \begin{align}
    \frac{\#(v, c)}{|\D|} &=  \frac{\sum_{r=1}^w (\#(v, c)_{\rar} + \#(v, c)_{\lar})}{\sum_{r=1}^w (|\D_{\rar}| + |\D_{\lar}|)} = \frac{1}{2w} \sum_{r=1}^w \left( \frac{\#(v, c)}{|\D_{\rar}|} + \frac{\#(v, c)}{|\D_{\lar}|} \right)\\
    &\overset{p}{\to} \frac{1}{r}\sum_{r=1}^{w} \frac{d_v}{2|E|}(P^r)_{v, c}
  \end{align}
  where the second equality uses the fact that $|\D_{\rar}| = |\D_{\lar}| =
  \frac{|\D|}{2w}$ and the convergence comes from applying [Theorem XX],
  together with the continuous mapping theorem.
\end{proof}

This gives everything we need to prove the main theorem of this section, that
allows us to understand the matrix that DeepWalk is implicitely factorising.

\MainDeepWalk*

\begin{proof}
  Firstly, by summing the result of [Theorem XX] we get
  \begin{align*}
    \frac{\#(w)}{|\D|} &= \sum_{c \in V}\#(w, c)\\
                       &\overset{p}{\to} \sum_{c \in V} \frac{1}{w}\sum_{r=1}^w \pi_v(P^r)_{v, c}\\
                       &\frac{\pi_v}{w}\sum_{r = 1}^w \sum_{c \in V}(P^r)_{v,c} = \frac{\pi_v}{w}\sum_{r = 1}^w 1 = \pi_v
  \end{align*}
  since $P^r$ is stochastic for any $r$.\\
  Similarly, using the fact that $\pi$ is in detailed balance with $P$, and thus
  the result of [Theorem XX] can be rewritten as
  \[\frac{\#(v, c)}{|\D|} \overset{p}{\to} \frac{1}{w} \sum_{r = 1}^w \pi_c
    (P^r)_{c,v}\]
  it can be shown that $\frac{\#(c)}{|\D|} \overset{p}{\to} \pi_c$.\\
  Using this and by applying the continuous mapping theorem
  \begin{align*}
    \frac{\#(v, c) \cdot |\D|}{\#(v) \cdot \#(c)} = \frac{\frac{\#(v,c)}{|\D|}}{\frac{\#(v)}{|\D|} \cdot \frac{\#(c)}{|\D|}} & \overset{p}{\to} \frac{\frac{1}{w}\sum_{r=1}^{w} \frac{d_v}{2|E|}(P^r)_{v, c}}{\frac{d_v}{2|E|} \cdot \frac{d_c}{2|E|}}\\
                                                                                                                             &=\frac{2|E|}{w}\sum_{r=1}^w (P^r)_{v,c} \frac{1}{d_c} = \frac{2|E|}{w}\left( \sum_{r=1}^w(P^rD^{-1})_{v,c} \right)
  \end{align*}
  where the last equality follows since $D$ is diagonal.\\

  From this, applying [Theorem XX] gives that as $t \to \infty$ DeepWalk is equivalent to
  factorising
  \[\log \left( \frac{2|E|}{w}\left( \sum_{r=1}^w P^r \right)D^{-1}\right) - \log{b}\]
\end{proof}

This proof gives a better understanding of what the DeepWalk algorithm is doing.
There are many other good questions to ask about DeepWalk, for example whether
the algorithm is garunteed to converge under stochastic gradient descent, but we
will not explore this any further in this essay.

*** Here can talk about how long it takes to tend towards this, potentially talk
briefly on convergence of stochastic gradient descent etc. Important point, how
large does $t$ need to be for this to be meaningful? ***

*** Can I try and use the above result to show that DeepWalk is garunteed to
converge, in other words that the objective function is convex ***

\section{Matters of Convergence}

*** This section is unlikely to be included ***
\section{DeepWalk Sucks}
Whilst the original DeepWalk paper made a huge impact on the future development
of social representation learning on graphs, the algorithm itself has a number
of drawbacks that have been pointed out in subsequent papers. In this short
section, I will detail some of the issues and criticism of the implementation of DeepWalk.\\

Firstly, as has briefly been mentioned, in the original paper Hierarchical
Softmax was used to estimate the probabilities in the softmax layer of SkipGram.
This is important, since to calculate the partition function (denominator of the
softmax probability) would normally have complexity $O(|V|)$. Using Hierarchical
softmax however, the nodes are asssigned to the leaves of a binary tree. This
reduces the computational complexity to $O(\log{|V|})$.\\

Negative Sampling on the other hand is developed ...

*** Talk about Node2Vec and improvements to DeepWalk etc. Talk about various
shortcomings, just look at papers that came afterwards that shred DeepWalk.
Point to the algorithms that improved it.

Plan: Pick 2 or 3 algorithms, state what they improved, link to the papers.
***
\section{Dynamic DeepWalking}
In this section we transition away from the static implementation of DeepWalk
towards network embeddings for Dynamic datasets. In particular, the focus of the
section is to introduce an adaptation of DeepWalk to dynamic datasets as
proposed in the paper published by Sajjad et al.\cite{sajjad2019} in early
2019.\\

As stated in the motivation for the paper, most of the recently developed
representation learning methods can only be applied to static graphs while many
real-world graphs are constantly changing over time. Therefore, to apply these
methods the graph must be reanalysed at regular snapshots in time. This is very
inneficient since in a given epoch, it is unlikely that the graph structure will
change dramatically. Sajjad et al. proposed a modified version of the DeepWalk
algorithm that is better suited to dynamic graphs and utilises the fact that the
graph structure is unlikely to change significantly to develop a much more
efficient way of analyzing social representation on dynamic graphs.
The computational complexity of the algorithm depends on the graph density and
number of edges added per epoch, however in the case of social networks these
are both low and so the alggorithm is well suited to this application.


\section{Discussion of an application}

The aim of this section is to bring the theory discussed on Dynamic DeepWalking
into practice.

\bibliography{references}
\bibliographystyle{ieeetr}



\printindex
\end{document}